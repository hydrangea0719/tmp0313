\newpage
\changeindent{0cm}
\begin{thebibliography}{99}
\changeindent{2cm}



% irony 1
\bibitem {GIBBS1991523}
Raymond W. Gibbs and Jennifer O’Brien. 
Psychological aspects of irony understanding. 
Journal of Pragmatics, Vol. 16, No. 6, pp. 523--530, 1991.


% irony 2
\bibitem {HAVERKATE199077}
Henk Haverkate. 
A speech act analysis of irony. 
Journal of Pragmatics, Vol. 14, No. 1, pp. 77--109, 1990.


% irony 3
\bibitem {Ozerova} 
オーゼロヴァ・アナスタシーア.
言語的皮肉の現象についての理論とその原理 : 日本語における皮肉の分析を中心に.
日本語・日本文化研究, Vol. 26, pp. 147--157, 2016.

% twitter 1
% 
\bibitem{DBLP:conf/emnlp/RiloffQSSGH13}
Ellen Riloff, Ashequl Qadir, Prafulla Surve, Lalindra De Silva, Nathan Gilbert, and Ruihong Huang. 
Sarcasm as contrast between a positive sentiment and negative situation. 
In EMNLP, pp. 704--714, 2013.


% twitter 2
\bibitem{ghosh-veale-2017-magnets}
Aniruddha Ghosh and Tony Veale. 
Magnets for sarcasm: Making sarcasm detection timely, contextual and very personal. 
In Proceedings of the 2017 Conference on Empirical Methods in Natural Language Processing, pp. 482--491, Copenhagen, Denmark, September 2017. Association for Computational Linguistics.


% news headlines dataset 
\bibitem {misra2019sarcasm}
Rishabh Misra and Prahal Arora. 
Sarcasm detection using hybrid neural network. 
2019.


% SARC データセット
% https://aclanthology.org/L18-1102/
\bibitem {khodak2018}
Mikhail Khodak, Nikunj Saunshi, and Kiran Vodrahalli.
A Large Self-Annotated Corpus for Sarcasm.
In Proceedings of the Eleventh International Conference on Language Resources and Evaluation (LREC 2018), Miyazaki, Japan, May, 2018. European Language Resources Association (ELRA).


% CASCADE 
% https://aclanthology.org/C18-1156/
\bibitem {hazarika-etal-2018-cascade}
Devamanyu Hazarika, Soujanya Poria, Sruthi Gorantla, Erik Cambria, Roger Zimmermann, and Rada Mihalcea.
CASCADE: Contextual Sarcasm Detection in Online Discussion Forums.
In Proceedings of the 27th International Conference on Computational Linguistics, pp. 1837--1848, Santa Fe, New Mexico, USA, August 2018. Association for Computational Linguistics.


% Deep and Dense Sarcasm Detection
% https://arxiv.org/abs/1911.07474
\bibitem {pelser2019deep}
Devin Pelser and Hugh Murrell.
Deep and dense sarcasm detection.
2019.



% BERT 
% https://aclanthology.org/L18-1102
\bibitem {devlin-etal-2019-bert}
Jacob Devlin, Ming-Wei Chang, Kenton Lee, and Kristina Toutanova.
BERT: Pre-training of Deep Bidirectional Transformers for Language Understanding.
In Proceedings of the 2019 Conference of the North American Chapter of the Association for Computational Linguistics: Human Language Technologies, Volume 1 (Long and Short Papers), pp. 4171--4186, Minneapolis, Minnesota, June 2019. Association for Computational Linguistics.


% Transformer
% https://papers.nips.cc/paper/2017/file/3f5ee243547dee91fbd053c1c4a845aa-Paper.pdf
\bibitem {NIPS2017_3f5ee243}
Ashish Vaswani, Noam Shazeer, Niki Parmar, Jakob Uszkoreit, Llion Jones, Aidan N Gomez, \L ukasz Kaiser, and Illia Polosukhin.
Attention is all you need.
2017.
In Advances in Neural Information Processing Systems, Vol. 30.


% BookCorpus
% https://www.computer.org/csdl/proceedings-article/iccv/2015/8391a019/12OmNro0HYa
\bibitem {bookcorpus}
Yukun Zhu, Ryan Kiros, Rich Zemel, Ruslan Salakhut- dinov, Raquel Urtasun, Antonio Torralba, and Sanja Fidler.
Aligning books and movies: Towards story-like visual explanations by watching movies and reading books. 
In Proceedings of the IEEE international conference on computer vision (ICCV 2015), pp 19--27, Los Alamitos, CA, USA, December 2015. IEEE Computer Society.


% WordPiece
% https://arxiv.org/abs/1609.08144
\bibitem {wu2016googles}
Yonghui Wu, Mike Schuster, Zhifeng Chen, Quoc V. Le, Mohammad Norouzi, Wolfgang Macherey, Maxim Krikun, Yuan Cao, Qin Gao, Klaus Macherey, Jeff Klingner, Apurva Shah, Melvin Johnson, Xiaobing Liu, \L ukasz Kaiser, Stephan Gouws, Yoshikiyo Kato, Taku Kudo, Hideto Kazawa, Keith Stevens, George Kurian, Nishant Patil, Wei Wang, Cliff Young, Jason Smith, Jason Riesa, Alex Rudnick, Oriol Vinyals, Greg Corrado, Macduff Hughes, and Jeffrey Dean.
Google's Neural Machine Translation System: Bridging the Gap between Human and Machine Translation.
2016.


% Optuna
% https://github.com/optuna/optuna
\bibitem {akiba2019optuna}
Takuya Akiba, Shotaro Sano, Toshihiko Yanase, Takeru Ohta, and Masanori Koyama.
Optuna: A Next-generation Hyperparameter Optimization Framework.
In Proceedings of the 25th ACM SIGKDD International Conference on Knowledge Discovery \& Data Mining, KDD ’19, pp. 2623--2631, New York, NY, USA, 2019. Association for Computing Machinery.


% t-SNE
% 
\bibitem {JMLR:v9:vandermaaten08a}
Laurens van der Maaten and Geoffrey Hinton. 
Visualizing Data using t-SNE. 
Journal of Machine Learning Research, Vol. 9, No. 86, pp. 2579--2605, 2008.





\end{thebibliography}

% 参考文献(thebibliography)については iepaper.sty の line 112 ~ 116 で定義されている.
% README 読む分には,bibitem で書くみたい.
% 1つ1つ手打ちする必要があるので,卒論の参考文献のフォーマット等があるかどうか調べておく

