%\newpage
%\changeindent{0cm}
%\section{要素技術}
%\changeindent{2cm}


%%%%%%%%%%%%%%%%%%%%%%%%%%%%%%
% 3.1. BERT
% 3.2 SARC データセット
%%%%%%%%%%%%%%%%%%%%%%%%%%%%%%
\subsection{SARC データセット}
本研究では,皮肉の研究や皮肉検出システムの学習・評価のための大規模コーパスである,自己注釈付き Reddit コーパス(Self-Annotated Reddit Corpus; SARC)\cite{khodak2018} を使用した.
このデータセットはソーシャルメディアサイト Reddit に投稿されたコメントから構築されている.
図 \ref{fig:reddit_sample} に Reddit に投稿されたコメントの例を示す.
図の上のコメントは投稿元であり,下のコメントはその投稿元に反応したものである.
\par
図 \ref{fig:30_dataset_sample} に SARC データセットに含まれるコメントの例を示す.
各コメントには文章だけではなく,投稿者の識別子,subreddit と呼ばれる投稿トピック,ユーザ評価(good/bad),投稿日時,親投稿の情報も付与されている.
親投稿とは,最初の投稿から各コメントに至るまでの一連の投稿であり,全てのコメントは 1 つ以上の親投稿を保持している.
SARC データセットはこのようなやりとりのうち,末端のコメントのみに皮肉か非皮肉かのラベルが付与されている.
\par

%%% figure Reddit の投稿例
\begin{figure}[b]
 	\begin{center}
		\includegraphics[width=0.8\linewidth]{./figure/30_reddit.png}
		\caption{Reddit に投稿されたコメントの例(文献 \cite{khodak2018} Figure 1 より引用)}
		\label{fig:reddit_sample}
	\end{center}
\end{figure}
% end figure

%%% figure データの例
\begin{figure}[tb]
 	\begin{center}
		\includegraphics[width=0.8\linewidth]{./figure/30_dataset_sample.png}
		\caption{SARC データセットのデータの例}
		\label{fig:30_dataset_sample}
	\end{center}
\end{figure}
% end figure

Reddit へ投稿する際,ユーザは自分のコメントが皮肉を意図していることを明示する記号 “/s” を使用する.
このことを利用し,投稿コメントに “/s” を含むことを必要条件としてデータに皮肉ラベルが付けられている.
すなわち SARC データセットは,投稿者が皮肉を意図しているかどうかを判断基準としてラベル付けしていると言える.
ただしこの記号は,データセットに含まれる文章からは除去されている.
\par
またこのデータセットには,全ての subreddit からデータを抽出した main データセットと,politics の subreddit からデータを抽出した pol データセットがある.
それぞれに対して均衡データセット(balanced)と不均衡データセット(unbalanced)が用意されており,全部で 4 つのデータセットが用意されている.
均衡データセットには,同じ親投稿に対して 2 つのコメントが存在する.
そして 2 つのコメントの内訳は,皮肉データと非皮肉データが 1 つずつである.
本研究では SARC 2.0 main balanced \footnote{https://nlp.cs.princeton.edu/SARC/2.0/main/} データセットを使用した.
このデータセットに含まれる訓練データは 257,082 件,テストデータは 64,666 件である.
\par
次に,使用したデータセットの subreddit の情報について述べる.訓練データに含まれる subreddit は 4,902 種類,テストデータに含まれる subreddit は 2,666 種類,訓練データとテストデータに共通する subreddit は 2,194 種類であった.
表 \ref{tb:1_subreddit_data} に subreddit ごとのデータ数を示す.
データ数が最も多い subredit は politics で,訓練データ 13,668 件,テストデータ 3,406 件であった.
またデータ数が最も少ない subreddit は複数あり,データ数は 2 件であった.


%%% table subreddit の分布
\begin{table}[tb]
  \caption{subreddit ごとのデータ数の分布}
  \label{tb:1_subreddit_data}
  \centering
  \begin{tabular}{c c c c c} \hline

\multicolumn{3}{c}{データ (件)} & \multicolumn{2}{c}{subreddit (種類)} \\ \cline{4-5}
\multicolumn{3}{c}{(以上) - (未満)} & 訓練 & テスト \\ \hline

& - & 100 & 4,575 & 2,567 \\
100 & - & 500 & 248 & 77 \\
500 & - & 1,000 & 40 & 13 \\
1,000 & - & 2,000 & 17 & 6 \\
2,000 & - & 3,000 & 6 & 1 \\ 
3,000 & - & 4,000 & 7 & 2 \\
4,000 & - & 5,000 & 4 & 0 \\
5,000 & - & 6,000 & 1 & 0 \\
6,000 & - & 7,000 & 0 & 0 \\
7,000 & - & 8,000 & 1 & 0 \\
8,000 & - & 9,000 & 0 & 0 \\
9,000 & - & 10,000 & 1 &  0 \\
10,000 & - & & 2 & 0 \\ \hline
\multicolumn{3}{c}{合計} & 4,902 & 2,666 \\ \hline 

  \end{tabular}
\end{table}
% end





