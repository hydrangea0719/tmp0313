\newpage
\changeindent{0cm}
\section{関連研究}
\changeindent{2cm}



本章では皮肉表現の心理的・言語的側面の研究と,計算機による皮肉推定に関連する研究を紹介する.
\par

%%%%%%%%%%%%%%%%%%%%%%%%%%%%%%
\subsection{皮肉表現の心理的・言語的研究}

人間にとって,ある発話や文章が皮肉であることを理解するのは容易であるが,その現象を説明することは難しい \cite{GIBBS1991523}.
例えば,相手がその文章によって伝えたいことを理解する際に,その文章に皮肉表現が含まれていることを認識する必要はないことから,人間は無意識に皮肉を理解できると言える.
さらに話し手や書き手が皮肉を意図していなくても,受け取る側によって皮肉であると解釈されることがある.
これらは皮肉表現の心理的側面に焦点を当てた説明である.
言語的側面からは,皮肉表現は語用論的原則の違反によって説明されることがある \cite{HAVERKATE199077}.
これは皮肉表現が文章の文字通りの意味と書き手の意図とが異なる言語表現であるという説明を含んでいる.
以上のように皮肉表現は多角的に研究されているが,全ての皮肉表現を説明しきる理論は未だ存在しない \cite{Ozerova}.


%%%%%%%%%%%%%%%%%%%%%%%%%%%%%%
\subsection{皮肉推定のためのデータセット}

皮肉推定の研究に利用されているデータセットにはソーシャルネットワーキングサービスに投稿された文章を収集したものが多い.
特に Twitter \footnote{\url{https://about.twitter.com/en}} に投稿された文章(ツイート)を収集し,皮肉推定に使用している研究は多く見られる.
Riloff らは皮肉を明示するハッシュタグ “\#sarcasm ”,“\#sarcastic” を手がかりに皮肉データを収集した \cite{DBLP:conf/emnlp/RiloffQSSGH13}.
一方で Ghosh \& Veale はハッシュタグだけを手がかりとするとタグ付けされていないデータを収集できないことを指摘している \cite{ghosh-veale-2017-magnets}.彼らは Twitterbot を用いて,皮肉と思われるツイートに対して投稿者に直接確認を取ることで皮肉データを収集した.
\par
News Headlines Dataset For Sarcasm Detection \cite{misra2019sarcasm} は,ウェブニュースサイト TheOnion \footnote{\url{https://www.theonion.com}} と HuffPost \footnote{\url{https://www.huffpost.com}} からヘッドラインを収集したデータセットである.
TheOnion は時事問題に対して皮肉的なヘッドラインを付けることで有名である.
そのため TheOnion から収集したヘッドラインデータに皮肉ラベルを,HuffPost から収集したデータに非皮肉ラベルを付与している.
ヘッドラインのみのデータであるため各データに文脈が存在せず,1 つの文章で理解可能な皮肉表現となっている.



%%%%%%%%%%%%%%%%%%%%%%%%%%%%%%
\subsection{SARC データセットを用いた皮肉推定}
本研究では SARC データセット \cite{khodak2018} を用いて皮肉推定をした.
このデータセットの特徴として Twitter データセットやヘッドラインデータセットと比べて大規模であることが挙げられる.
またメタデータも充実しており,近年の皮肉表現に関する研究に大きく貢献しているデータセットである.
本節ではこのデータセットを用いた関連研究を紹介し,SARC データセットについては後の章で詳述する.
\par


%\par
%%%%%%%%%%%%%%%%%%%%%%%%%%%%%%
% 2018 05 CASCADE: Contextual Sarcasm Detection in Online Discussion Forums
2018 年に,ソーシャルメディアサイトへの投稿文章に対する皮肉推定のモデルとして CASCADE \cite{hazarika-etal-2018-cascade} が提案され,state-of-the-art を達成した.
このモデルは文章の情報に加えて,投稿者と投稿トピックを元にした情報を特徴量として利用する.
具体的には,Convolutional Neural Network (CNN) を用いて取得した文章の分散表現 $\overrightarrow{c}_{i,j}$,投稿者の過去の投稿文章から取得した分散表現 $\overrightarrow{u}_{i}$,投稿トピックにおける投稿文章から取得した分散表現 $\overrightarrow{t}_{j}$ を連結して分類に用いている.
結果として,文章の情報と投稿者や投稿トピックを元にした情報を併用することで,皮肉推定の精度が向上することを示した.


\par
%%%%%%%%%%%%%%%%%%%%%%%%%%%%%%
% 2019 Deep and Dense Sarcasm Detection
文章の情報のみを使用した研究もされている.
2019 年に Pelser \& Murrell \cite{pelser2019deep} は,投稿者などの情報はプライバシー設定やデータ欠損により常に利用可能とは限らないことを指摘し,56 層の深層ネットワークによる文章のみを利用した推定手法を提案した.
結果として,上述した CASCADE の性能を上回ることはできなかったものの,文章以外の情報を利用した他の既存手法に匹敵する性能を示した.


\par
%%%%%%%%%%%%%%%%%%%%%%%%%%%%%%
本研究では文章情報からの皮肉推定に重きを置くため投稿者の情報は使用しない.
また subreddit の情報を使用することで話題ごとの皮肉推定の有効性を確認する.







