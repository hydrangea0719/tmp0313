%\newpage
%\changeindent{0cm}
%\section{要素技術}
%\changeindent{2cm}


%%%%%%%%%%%%%%%%%%%%%%%%%%%%%%
% 3.1. BERT
% 3.2 Optuna
% 3.3 SARC データセット
%%%%%%%%%%%%%%%%%%%%%%%%%%%%%%
\subsection{Optuna}
% https://www.preferred.jp/ja/projects/optuna/
% https://github.com/optuna/optuna
% https://arxiv.org/pdf/1907.10902.pdf
% https://tech.preferred.jp/ja/blog/optuna-release/


Optuna \cite{akiba2019optuna} は,オープンソースのハイパーパラメータ自動最適化フレームワークである.
Tree-structured Parzen Estimator というベイズ最適化アルゴリズムを用いて,過去の試行に基づいて有望そうな領域を推定し再度試行する.
これを繰り返すことで最適なハイパーパラメータの値を自動的に発見する.
使用する際は目的関数を定め,その値がより大きくまたは小さくなるように推定を進める.
Optuna の主な特徴として,Define-and-Run スタイルの API,学習曲線を用いた試行の枝刈り,並列分散最適化が挙げられる.

